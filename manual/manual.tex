\documentclass[12pt,letterpaper]{article}

\title{Ornithology PI}

%% To translate this poster, change the language 
%% option for both babel and translator package.
%% (Don't worry when you get an 'unknown language'
%% error; just click 'recompile from scratch' at the
%% bottom of the error message window.
\usepackage[english]{babel}
\usepackage[english]{translator}

%% Then add your own OverleafPoster-<lang>.dict file
%% and provide your translations, based on the 
%% default OverleafPoster-English.dict.
\usedictionary{OverleafPoster}

%% For example, change "english" to "ngerman", 
%% for babel and translator above, and see what
%% happens. ;-) 
%% Note that the provided OverleafPoster-German.dict
%% contains dummy data!

\usepackage[utf8]{inputenc}
\usepackage[T1]{fontenc}
\usepackage[sfdefault,lf]{carlito}

\usepackage[margin=1.1cm,letterpaper]{geometry}
\usepackage{multicol}
\usepackage{ragged2e}
\usepackage{tabularx}
\usepackage[table]{xcolor}
\usepackage{graphicx}
\usepackage{fontawesome5}
\usepackage{qrcode}

\definecolor{OverleafGreen}{HTML}{4F9C45}
\graphicspath{{images/}}
\pagestyle{empty}
\RaggedRight
\parskip=12pt plus 4pt

\begin{document}

{\fontsize{40pt}{42pt}\bfseries\selectfont\color{OverleafGreen}%
\translate{Ornithology PI}%
\par}

\begin{multicols}{2}

{\LARGE\bfseries%
\mbox{\faIcon{github} chriamue/ornithology-pi\par}%
}
\par

{\Large%
\translate{}%
\par}

%\includegraphics[width=\linewidth]{OL_Editor_Screenshot}

\columnbreak
\setlength{\leftskip}{40pt}
{\centering%
%\includegraphics[width=.55\linewidth]{Overleaf_Challenge_Logo}%
\par}

% \vskip -\baselineskip

\translate{What Is It}\par
It's a device with a camera, that saves images if a bird is detected.
The species of the wird will be identified.

\end{multicols}

\vskip -\baselineskip

\hskip -1.1cm%
\renewcommand{\arraystretch}{1.8}%
\begin{tabularx}{\paperwidth}{%
   @{} p{1.5cm} @{}
   *4{>{\centering\arraybackslash\large\bfseries}X @{}}
   p{1.5cm} @{}
}
\rowcolor{gray!30}
\rule{0pt}{2.2cm} & 
 \faIcon{tree} & \faIcon{wifi} & \faIcon{tv} & \faIcon{crow} & \\[0.1cm]
%\includegraphics[height=1.8cm]{green_light_bulb} &
%\includegraphics[height=1.8cm]{projector_screen} &
%\includegraphics[height=1.8cm]{Globe_Mouse} &
%\includegraphics[height=1.8cm]{pencil-308509} & \\
& \translate{place} & \translate{connect} & \translate{watch} & \translate{identify} & \\
\end{tabularx}


\begin{large}
\faIcon{wifi} 
\translate{(1) Connect to Wifi} \par

Connect to wifi \textbf{ornithology-pi} using password \textbf{ornithology}.

\qrcode{
Network Name: ornithology-pi
Type: WPA
Password: ornithology
}

\faIcon{chrome}
\translate{(2) Visit Web Page}\par

Visit the url \textbf{http://10.42.0.1:8000}.

\qrcode{
http://10.42.0.1:8000
}

\faIcon{crow}
\translate{(3) Identify}\par

Now you can watch the livestream.

In the list the detected birds will be listed with tagged species.

\end{large}

\vfill

\includegraphics[width=2cm]{logo}
\hfill
{\LARGE\bfseries\itshape\color{OverleafGreen} https://github.com/chriamue/ornithology-pi}
\end{document}
